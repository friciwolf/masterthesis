\section*{Danksagung}

Hiermit möchte ich die Gelegenheit nutzen und mich bei allen bedanken, die auf direkte oder indirekte Art und Weise einen Beitrag zu dieser Arbeit geleistet haben.

In erster Linie möchte ich Prof. Dr. Hans Martin Erdmann danken, der die Entstehung dieser Arbeit ermöglicht hat. Seine kritischen Anmerkungen bei den Meetings waren stets enorm hilfreich. Ebenfalls Dank gebührt Prof. Dr. Alexander Schmidt für das Zweitgutachten dieser Arbeit.

Mein besonderer Dank gilt meiner Betreuerin Svenja Diekmann und meinem Betreuer Niclas Eich. In der Zusammenarbeit haben sie nicht nur tief- und übergreifendes Fachwissen und Sachverständnis bewiesen, sondern zeigten sich bei eventuellen Rückfragen geduldig und äußert motiviert, diese möglichst präzise zu beantworten. In ihrer Gesellschaft durfte ich in einer geselligen und motivierenden Atmosphäre arbeiten.

Ich möchte mich bei den Di-Higgs Kollegen Dennis Noll, Peter Fackeldey und Benjamin Fischer für ihre einfallsreichen Hinweise bedanken. An Stellen, wo uns Lösungsansätze fehlten, haben eure Ideen uns wieder auf den richtigen Weg gebracht.

Von den Kollegen aus der Auger-Gruppe habe ich ebenfalls viel Unterstützung bekommen. Namentlich Josina Schulte hat mir durch ihre hervorragende Einführung in conditional invertible neural networks nicht nur den notwendigen Startschub gegeben, sondern mich dabei auch mit erstklassiger Fachliteratur versorgt. Sie hatte immer ein offenes Ohr für Probleme und hat neue Ideen und Ansätzte kritisch hinterfragt.
Vielen Dank!