\Section{Conclusion}
\label{sec:conclusion}

Since the discovery of the Higgs boson in 2012, intensive studies on its properties and couplings to SM particles have been performed. In the SM, the gluon-induced ZH production processes due to their loop-induced nature at leading order are good candidates for SM testing measurements. As these processes cancel destructively in the SM, these are not directly accessible by experiment. For this reason, the quark induced WH and ZH production cross sections are encoded in the BSM sensitive observable $R$, for which systematic uncertainties cancel.

In this thesis, inference with cINNs on the signal strength modifiers for the ggZH, Drell-Yan-like ZH and WH processes has been performed. These networks can perform inference several orders of magnitude faster than likelihood fits through a continuous, differentiable and invertible model. The trained network model is capable of reconstructing the three signal strength modifier and 14 nuisance parameters.
For the Standard Model expectation the obtained values for the signal processes were

\begin{equation*}
		\mu_\text{ggZH} = 5.10^{+3.57}_{-3.50} \, \, \quad \mu_\text{ZHDY} = 3.90^{+2.62}_{-2.57} \, \, \quad \mu_\text{WH} = 3.02^{+1.92}_{-1.90}
\end{equation*}

The uncertainty on these values can be further improved with increased analysis sensitivity. For the three signal processes, the network model suffers from a signal sensitivity drop in the region $\mu\lesssim10$. This effect has been observed in terms of the resulting posteriors, which have a broader shape in this region.

The calibration curves of the cINN model show the absence of internal biases. The latent space distribution follows the expected normal distribution $\mathcal{N}(z | 0,1)$. The loss function indicates stable training. All of these effects point to a good approximation of the true posteriors. The training dataset includes the treatment of statistical and systematic uncertainties. For the shape-changing uncertainties, histogram morphing has been applied.

Network sensitivity develops from the mean of the priors. The nuisance parameters can be grouped into well-reconstructed, weakly-reconstructed and unrecognized nuisance parameters. For the well-reconstructed nuisance parameters, the posteriors take a narrow shape and the predictions scatter closely around the true MC value. For the unrecognized nuisance parameters, the network returns the prior distribution as the network developed no sensitivity to these parameters. For the weakly reconstructed nuisance parameters, the posteriors take a similar shape as the priors and the incomplete sensitivity results in a lense-like shape in the prediction-expectation distribution.

Further studies can performed in grouping some nuisance parameter inputs and by studying the network's performance with respect to these grouped inputs. With improved analysis techniques, the predictions on the SM expectations can be further improved. The study on the propagation of observable gradients through the network is also promising.