\Section{Conclusion}
\label{sec:conclusion}

Since the discovery of the Higgs boson in 2012, intensive studies on its properties and couplings to SM particles have been performed. In the SM, the gluon-induced ZH production process is a good candidate for SM testing measurements due to its loop-induced nature at leading order. As the contributing diagrams interfere destructively in the SM, the expected process contribution is low for a measurement. For this reason, the quark induced WH and ZH production cross sections are also taken into account.

In this thesis, inference with cINNs on the signal strength parameters for the ggZH, Drell-Yan-like ZH and WH processes has been performed. The networks can perform inference several orders of magnitude faster than likelihood fits through a continuous, differentiable and invertible model. The trained model is capable of reconstructing the three signal strength modifier and 14 nuisance parameters. The datasets already include the treatment of statistical and systematic uncertainties. For the shape-changing uncertainties, histogram morphing has been applied.

The absence of internal biases is shown by the calibration curves of the cINN model. The latent space distributions follow the expected normal distribution $\mathcal{N}(z | 0,1)$. The loss function indicates stable training. All of these effects imply a good approximation of the true posteriors.

The network sensitivity develops from the mean of the priors. The nuisance parameters can be grouped into well-reconstructed, weakly-reconstructed and unrecognized nuisance parameters. For the well-reconstructed nuisance parameters, the posteriors take a narrow shape and the predictions scatter closely around the true MC value. For the unrecognized nuisance parameters, the network returns the prior distribution as the network developed no sensitivity to these parameters. For the weakly reconstructed nuisance parameters, the posteriors take a similar shape as the priors and the increased sensitivity results in a broader shape in the prediction-expectation distribution.

For the Standard Model expectation, where $\mu=1$ for each signal process, the obtained values for the signal processes were

\begin{equation*}
	\mu_\text{ggZH} = 5.10^{+3.57}_{-3.50} \, \, \quad \mu_\text{ZHDY} = 3.90^{+2.62}_{-2.57} \, \, \quad \mu_\text{WH} = 3.02^{+1.92}_{-1.90}.
\end{equation*}

For the three signal processes, the network model suffers from a signal sensitivity drop in the region $\mu\lesssim10$. This effect has been observed in terms of the resulting posteriors, which have a broader shape in this region. The uncertainty on these values can be further improved with increased analysis sensitivity. A study on the propagation of observable gradients through the network is also promising.