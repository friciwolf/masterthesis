\Section{Experimental Setup}

In this chapter the experimental context of this thesis will be discussed. 

\Subsection{The Large Hadron Collider}
\label{sec:theory}

The Large Hadron Collider (LHC) is currently the most powerful particle accelerator in the world. Hosted at CERN in Geneva at the Swiss-French border and first put into operation on $\text{10}^\text{th}$ September 2008, the LHC is designed for proton and heavy lead ion collisions. The machine has gone through several upgrades between the consecutive data-taking phases (Runs) called Long Shutdowns (LS). During these the proton beam energy has been gradually increased from \SI{3.5}{\tera\electronvolt} to a recently -- on the $\text{5}^{\text{th}}$ July, 2022 to be precise -- achieved energy of \SI{6.8}{\tera\electronvolt} \cite{Alici:2773265} resulting in a total centre-of-mass (CM) proton-proton collision energy of $\sqrt{s} = \SI{13.6}{\tera\electronvolt}$. Similarly, the beam intensity has seen an increase from $1.1 \times 10^{11}$ protons per bunch (ppb) and $\sim 200$ bunches to a projected $\sim 1.8 \times 10^{11}$ ppb and $\sim 2500$ bunches \cite{Fartoukh:2790409, Karastathis:2750302}. With a theoretical maximum CM energy of $\sqrt{s} = \SI{14}{\tera\electronvolt}$ and integrated luminosity of $L = \SI{10d34}{\centi\meter^{-2}\second^{-1}}$ it holds the record in these measures among concurring experiments. In order to achieve such luminosities, the beams are kept on a circular trajectory using superconducting NbTi magnets operating at \SI{1.9}{\kelvin} thanks to the superfluid helium bath at about \SI{0.13}{\mega\pascal} \cite{Brüning:782076}.

As a result of consecutive accelerator upgrades, today's LHC has an impressive and ever-growingly complex pre-accelerator structure as shown in fig. \ref{fig:lhcstructure}; consequently, the proton bunches first go through multiple preparation steps before they get injected into the \SI{27}{\kilo\meter} tunnel of the LHC where the four main experiments (ALICE, ATLAS, LHCb and CMS) and their interaction points are located. 

\begin{figure}[h!]
	\centering
	\includegraphics[width=0.9\textwidth]{figures/theoryexperiment/CCC-v2017}
	\caption{The (pre-) accelerator structure of the LHC \cite{Mobs:2197559}}
	\label{fig:lhcstructure}
\end{figure}


\Subsection{The Compact Muon Solenoid}

One of general purpose detectors at the Large Hadron Collider at CERN is the Compact Muon Solenoid (CMS).
