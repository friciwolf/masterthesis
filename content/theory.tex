\Section{The CMS Experiment at the LHC}
\label{sec:theory}

The Large Hadron Collider (LHC) is currently the most powerful particle accelerator in the world. With a recently -- on the $\text{7}^{\text{th}}$ July, 2022 to be precise -- achieved of a total centre-of-mass proton-proton collision energy of $\sqrt{s} = \SI{13.6}{\tera\electronvolt}$ for the upcoming Run3 data-taking period it holds the record in that measure among concurring experiments. The accelerator complex hosted at CERN in Geneva at the Swiss-French border.

As a result of consecutive accelerator upgrades, today's LHC has an impressive and ever-growingly complex pre-accelerator structure as shown in fig. \ref{fig:lhcstructure}.

\begin{figure}[h!]
	\centering
	\includegraphics[width=\linewidth]{figures/theoryexperiment/CCC-v2017}
	\caption{The (pre-) accelerator structure of the LHC \cite{Mobs:2197559}}
	\label{fig:lhcstructure}
\end{figure}



One of general purpose detectors at the Large Hadron Collider at CERN is the Compact Muon Solenoid (CMS).
