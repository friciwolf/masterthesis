\Section{Introduction}
\label{sec:introduction}

The discovery of the Higgs boson by the CMS and ATLAS experiments at the Large Hadron Collider in 2012 has opened new frontiers towards new physical phenomena \cite{Chatrchyan_2012}. As the Standard Model of Particle Physics (SM) is incomplete, precision measurements in the Higgs sector are promising candidates for Beyond the Standard Model model-limiting measurements. It has been proposed that the ratio of cross sections of the gluon fusion induced ZH final states to the quark induced ZH production is a sensitive observable towards such phenomena \cite{Harlander_2018}. As the direct extraction of the gluon fusion induced process is limited by current experimental setup and analysis techniques due to the destructive interference of the contributing diagrams, the quark induced WH production cross section is included in the proposed measurement.

Newly developed deep learning methods have shown great success in analytically unsolvable tasks such as image classification, generation and captioning, text translation and generation and regression tasks \cite{LeCun2015}. In a physics context, they have become a powerful tool for flavour identification, pileup mitigation, detector effect simulation and process classification \cite{Schwartz_2021}. Most of these modern approaches are built upon neural networks and tackle the challenge of Bayesian inference by data-driven methods.

Conditional invertible neural networks have already proven to be successful in guided image generation and -colorization \cite{cINN_im_gen} and stochastic model learning \cite{BayesFlow}. Thanks to their invertible nature, these networks can infer the posterior distributions directly. In physics, these networks have proven to be successful in their application to stellar parameter estimation \cite{Ksoll_2020}, cosmic ray source property inference \cite{Bister_2022} and detector effect simulation \cite{Bellagente_2020}.

The continuous growth in model complexity in high energy physics demands increasingly time-consuming model fits. In this thesis inference of signal strength modifier and nuisance parameters with cINNs will be discussed. Upon successful training, these networks can infer the physics parameters quickly and for low computational cost. The performance measures of the resulting network will be characterized and discussed in detail.